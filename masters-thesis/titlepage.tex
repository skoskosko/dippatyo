% Enable the use of @ character in command names.

\makeatletter

% The titles of the work. If there is no subtitle, leave the \myfisubtitle or
% \myensubtitle command arguments empty. Pass the title in the primary
% language as the first argument and its translation to the secondary language
% as the second.

\if@langenglish

    \title{\myentitle}{\myfititle}

\else

    \title{\myfititle}{\myentitle}

\fi

\if@langenglish

    \subtitle{\myensubtitle}{\myfisubtitle}

\else

    \subtitle{\myfisubtitle}{\myensubtitle}

\fi

% The author name.

\author{\myauthor}

% The examiner information. If your work has multiple examiners, replace with
%
%   \examiner[<label>]{<name> \\ <name>}
%
% where <label> is an appropriate (plural) label, e.g. Examiners or
% Tarkastajat, and <name>s are replaced by the examiner names, each on their
% separate line.

\examiner{\myexaminers}

% The finishing date of the thesis (YYYY-MM-DD).

\finishdate{\myyear}{\mymonth}{\myday}

% The type of the thesis (e.g. Kandidaatintyö or Master of Science Thesis) in
% the primary and the secondary languages of the thesis.

\if@langenglish

    \thesistype{\myenthesistype}{\myfithesistype}

\else

    \thesistype{\myfithesistype}{\myenthesistype}

\fi

% The faculty and degree programme names in the primary and the secondary
% languages of the thesis, respectively.

\if@langenglish

    \facultyname{\myenfacultyname}{\myfifacultyname}

\else

    \facultyname{\myfifacultyname}{\myenfacultyname}

\fi

\if@langenglish

    \programmename{\myenprogrammename}{\myfiprogrammename}

\else

    \programmename{\myfiprogrammename}{\myenprogrammename}

\fi

% The keywords of the thesis in the primary and the secondary languages of the
% thesis.

\if@langenglish

    \keywords{\myenkeywords}{\myfikeywords}

\else

    \keywords{\myfikeywords}{\myenkeywords}
\fi

% Make @ a regular letter again.

\makeatother

% Actually generate the title page based on the above commands.

\maketitle
