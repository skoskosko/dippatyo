Syvyyden analysointi on keskeinen osa monia käytännön sovelluksia.
Se on yksi ensimmäisistä askeleista, jotka mahdollistavat interaktion reaalimaailman ja tietokoneohjelmien välillä.
Yksi pitkään käytetty tapa tähän on stereo-analyysi,
jossa kahden kuvan perusteella pyritään arvioimaan kohteiden etäisyyttä kameroista.
Tämä saavutetaan tunnistamalla kuvista samankaltaisia alueita, jotka todennäköisesti viittaavat samaan kohteeseen.
Stereokuva-analyysi vaatii, että kahden kuvan perspektiivi on erilainen toisiinsa nähden.
Tällä hetkellä yksi käytetyimmistä algoritmeista on Hirschmüllerin kehittämä menetelmä \cite{hirschmuller2005babel}.

Stereokuva-analyysissä kuvan pikselit tai pikseliryhmät käydään läpi.
Suurissa kuvissa tämä prosessi voi käydä nopeasti hyvin raskaaksi.
Koneoppiminen on kuitenkin tarjonnut mahdollisen ratkaisun tähän ongelmaan \cite{LagaHamid2022ASoD}.
Vaikka itse tarvittava prosessointiteho ei välttämättä vähene, koneoppimisen avulla on mahdollista saavuttaa nopeampaa ja tehokkaampaa prosessointia.
Tämä johtuu pääasiassa nykyisten tietokonekomponenttien optimoinnista koneoppimista varten.
Koneoppimismallit on helppo ajaa suoraan näytönohjaimen raudalla, mikä mahdollistaa laitteiston optimoinnin paremmin käyttötarkoitukseen \cite{LeeVictorW2010Dt1G}.

Nykyisin tilan syvyyttä voidaan arvioida myös muilla tehokkaammilla menetelmillä, kuten lidarin avulla.
Tämä syvyyden analysointitekniikka on yleisesti käytössä esimerkiksi autojen automatiikassa sen luotettavuuden vuoksi verrattuna perinteiseen kuva-analyysiin \cite{RorizRicardo2022ALTA}.
LIDAR-data tarjoaa kuitenkin vähemmän mahdollisuuksia jälkikäteisanalyysiin.
Vaikka teoriassa LIDAR-datasta voidaan suorittaa samat analyysit ja segmentoinnit kuin kuvista \cite{SunJiaming2020DRS3}, se tarjoaa vain syvyysdatan, kun taas stereo kuvat jotka tarjoavat syvyyden lisäksi itse kuvadatan.

Kuvista on pitkään tehty erilaista objektin tunnistusta, ja erilaisten automaatiojärjestelmien yleistyessä on tämän ongelman merkitys vain kasvanut.
Objektin tunnistus kuvassa on siis koneoppimisen keskeisiä haasteita, ja tästä syystä aiheesta löytyy paljon tutkimusta.
Autonomisen liikenteen tutkimuksen yleistyessä kohteen tunnistuksen yhdistäminen 3D-dataan, olipa kyseessä LIDAR- tai stereokuva-analyysistä saatu data, on erityisen hyödyllistä kaupunkitilan analysoinnissa \cite{MengZeYu2024TODf}. Näiden tekniikoiden yhdistäminen voi tuottaa merkittäviä etuja erilaisissa oikeassa maailmassa toimivien järjestelmien kehityksessä ja käytössä.
Lidar data tai stereo data ei kuitenkaan ole välttämätöntä, vaan koneoppimisen avulla voidaan jopa 2D-kuvista hakea erilaista 3D-dataa \cite{MaXinzhu20243ODF}.


Stereokuva-analyysi ja objektin tunnistus ovat hyvin yleisiä tutkimuskohteita.
Niiden yhdistäminen rajoittuu kuitenkin usein 3D-datan avulla tapahtuvaan asioiden tunnistamiseen tai objektien mallintamiseen, eikä 3D-objektien poistamiseen, kuten tässä työssä tavoitellaan.
Vaikka perinteinen stereokuva-analyysi yhdistettynä kohteen tunnistukseen tuottaa tuloksia, ja sillä tavalla tuotettua dataa hyödynnetään tuotetun mallin koulutuksessa, neuroverkkomallien tapauksessa prosessointi on huomattavasti nopeampaa.
Tällöin voidaan tarkastella, miten malli toimii tilanteissa, joihin perinteinen generointi ei pysty.
