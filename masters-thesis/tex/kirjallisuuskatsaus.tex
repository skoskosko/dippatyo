\chapter{Kirjallisuuskatsaus}%
\label{ch:kirjallisuuskatsaus}

Syvyyden analysointi on keskeinen osa monia käytännön sovelluksia.
Se on yksi ensimmäisistä askeleista, jotka mahdollistavat interaktion reaalimaailman ja sellaisten tietokoneohjelmien välillä,
joissa kahden kuvan perusteella pyritään arvioimaan kohteiden etäisyyttä kameroista. 
Tämä saavutetaan tunnistamalla kuvista samankaltaisia alueita, jotka todennäköisesti viittaavat samaan kohteeseen.
Stereokuva-analyysi vaatii kahden kuvaperspektiivin olevan toisiinsa nähden erilaiset.
Ensimmäisissä tekniikoissa käytettiin kuvaprosessoinnin reunan tunnistusta, jonka avulla pyrittii löytämään stereokuvista syvyyttä \cite{BakerHenryHarlyn1982DfEa}.
Tämä tapa vaatii kuvalta esiprosessoinnin ja sen tarkkuus ei ole paras mahdollinen, koska se ei ota huomioon mitään muuta kuin suuria muutoksia.
Lisäksi tämä tekniikka joutuu käymään kuvista paljon pikseleitä läpi. 
Mikä johtaa suureen prosessointi aikaan, mikä puolestaan johtaa pidempään prosessointiaikaan.
Tällä hetkellä yksi käytetyimmistä algoritmeista on Hirschmüllerin kehittämä menetelmä \cite{hirschmuller2005babel},
jossa reunantunnistustapaa on parannettu, etsimällä samankaltaisia alueita. 
Hirschmüllerin mallissa on myös paljon optimiointeja, pyyhkäisyjen määrään ja etsintä alueisiin jotta prosessointi aika saataisiin minimoitua.

Stereokuva-analyysissä käydään läpi kuvan pikseleitä ja pikseliryhmiä.
Suurissa kuvissa tämä prosessi voi käydä nopeasti hyvin raskaaksi.
Koneoppiminen on kuitenkin tarjonnut mahdollisen ratkaisun tähän ongelmaan \cite{LagaHamid2022ASoD}.
Vaikka koneoppimisessa itse tarvittava prosessointiteho ei välttämättä vähene, 
sen avulla on mahdollista saavuttaa nopeampaa ja tehokkaampaa prosessointia,
johtuen pääasiassa nykyisten tietokonekomponenttien optimoinnista koneoppimista varten.
Koneoppimismallit on helppo ajaa suoraan näytönohjaimilla, mikä puolestaan mahdollistaa laitteiston optimoinnin paremmin käyttötarkoitukseen \cite{LeeVictorW2010Dt1G}.
Tehokkaammat tietokonekomponentit ovat johtaneet useiden eri teknologioiden siirtymisen hyödyntämään koneoppimista.
Varsinkin abstraktien ongelmien ollessa kyseessä koneoppimisella voidaan saada aikaan tehokkaita ratkaisuja.


Nykyisin tilan syvyyttä voidaan arvioida myös kuvapohjaista-analyysiä tehokkaammilla menetelmillä, kuten lidarin avulla.
Tämä syvyyden analysointitekniikka on yleisesti käytössä, esimerkiksi autojen adaptiivisissä nopeussäätimissä, koska se on perinteistä kuva-analyysiä luotettavampaa \cite{RorizRicardo2022ALTA}.
LIDAR-data tarjoaa kuitenkin vähemmän mahdollisuuksia jälkikäteisanalyysiin.
Vaikka teoriassa LIDAR-datasta voidaan suorittaa samat analyysit ja segmentoinnit kuin kuvista \cite{SunJiaming2020DRS3}, tarjoaa se vain syvyysdatan.
Stereokuvissa analysoidun syvyysdatan lisäksi on tarjolla myös värejä, tekstuureja ja muita mahdollisia lisäanalyysikohteita.

Tässä työssä hyödynnetään myös objektintunnistusta.
Kuvista on pitkään tehty erilaista objektin tunnistusta, ja erilaisten automaatiojärjestelmien yleistyessä on tämän ongelman merkitys vain kasvanut. 
Objektin tunnistus kuvassa on koneoppimisen keskeisiä haasteita, näin ollen aiheesta löytyy paljon tutkimustietoa.
Autonomisen liikenteen tutkimuksen yleistyessä kohteen tunnistuksen yhdistäminen 3D-dataan, olipa kyseessä LIDAR- tai stereokuva-analyysistä saatu data, on erityisen hyödyllistä kaupunkitilan analysoinnissa \cite{MengZeYu2024TODf}. Näiden tekniikoiden yhdistäminen voi tuottaa merkittäviä etuja erilaisissa oikeassa maailmassa toimivien järjestelmien kehityksessä ja käytössä.
Lidar-data tai stereodata eivät kuitenkaan ole välttämättömiä, vaan koneoppimisen avulla voidaan jopa 2D-kuvista hakea erilaista 3D-dataa \cite{MaXinzhu20243ODF}.


Stereokuva-analyysi ja objektintunnistus ovat hyvin yleisiä tutkimuskohteita.
Niiden yhdistäminen rajoittuu kuitenkin usein 3D-datan avulla tapahtuvaan asioiden tunnistamiseen tai objektien mallintamiseen, eikä 3D-objektien poistamiseen, kuten tässä työssä tavoitellaan.
Vaikka perinteinen stereokuva-analyysi yhdistettynä kohteen tunnistukseen tuottaa tuloksia, ja sillä tavalla tuotettua dataa hyödynnetään mallin koulutuksessa, neuroverkkomallien tapauksessa prosessointi on huomattavasti nopeampaa.
Lisäksi yksi malli poistaa tarpeen useille välivaiheille, joita koulutusdatan käsittelyyn muutoin vaaditaan.
Mallia voi myös hyödyntää tilanteissa joissa datan generointi kuvasta ei muutoin ole mahdollista, esimerkiksi puutteellisesta datasta johtuen.
