
Koneoppimisessa datan hankinta on usein yksi suurimpia haasteita. 
Tämä työ käsittelee mallin luomista, jonka avulla voidaan tunnistaa stereokuvasta sen syvyys jättäen huomiotta irtonaiset objektit.
Koska tälläista koulutusdataa ei ole helposti saatavilla yritetään se luoda yleisemmin saatavilla olevan datan avulla. 

Koneoppimista käytetään usein objektien tunnistamiseen ja segmentointiin. Tästä johtuen siihen käytettäviä datamalleja on runsaasti saatavilla.

Toinen tarvittava tieto on syvyysdata. 
Stereoanalyysiä on tehty jo ennen koneoppimisen yleistämistä joten sitäkin varten olevia datasettejä on runsaasti saatavilla. 

Puuttuvaksi dataksi jää kuvastatunnistettujen kohteiden taakse jäävä syvyys. Tähän ei ole helposti saatavilla olevia datasettejä.
Näin ollen koulutusdataan on pyritty lisäämään kohteiden takana oleva syvyys arvioimalla niiden ympärillä olevien syvyyksien perusteella.
Tässä on käytetty hyödyksi arvioituja syvyyksiä kuvan eri alueilla ja pyyhkäisty syvyys lineaarisesti kohti tätä arvioitua syvempää kohtaa.

Koska lopullinen koulutusdata on hyvin likimääräistä ei voida olettaa että saatu malli on tarkka.
Kuitenkin jo huonosti toimivaa matalan resoluution mallia voidaan käyttää arvioimaan,
pystyisikö koneoppimista hyödyntää tälläisen "huonolaatuisen" datan kanssa hyödyllisen lopputuloksen saavuttamiseen.