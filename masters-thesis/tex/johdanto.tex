\chapter{Johdanto}%
\label{ch:johdanto}

Koneoppimisella voidaan ratkaista monenlaisia ongelmia huolimatta siitä, että se on aina rajattu saatavilla olevaan koulutusdataan.  
Koneoppimisen avulla voidaan pyrkiä ratkaisemaan normaalisti erittäin monimutkaisia algoritmeja vaativia ongelmia.  
Käytännössä mikä tahansa tietotekninen ongelma on teoriassa ratkaistavissa koneoppimisella,  
vaikkakaan se ei aina ole tarpeellinen tai mahdollinen ratkaisu.  
Tarpeettoman monimutkaisuuden sekä ”black box” -ongelmien välttelyn ohella  
yhtenä suurimpana esteenä erilaisten koneoppimismallien toteuttamisessa on hyvän koulutusdatan hankkiminen.  
Suurin osa koneoppimisesta perustuu koulutusdataan, vahvistusoppiminen yhtenä yleisimpänä poikkeuksena \cite{alma9911523590705973}.  
Koulutusdatan käyttö kuitenkin ohjaa lopullisen mallin tason vahvasti sidonnaiseksi koulutusdatan laatuun.  

Mikäli siis halutaan luoda malli, joka ei käytä yleisesti saatavilla olevaa koulutusdataa,  
hankaloituu koulutusprosessi suuresti.
Yleisimmin saatavilla olevat konenäkömallit ovat tarkoitettu ihmisten sekä asioiden tunnistamiseen \cite{kagglecomvis}.  
Jos tavoittelemamme asia poikkeaa yleisestä tunnistusongelmasta ja emme halua tuottaa uutta dataa tyhjästä, 
jää mahdolliseksi vaihtoehdoksi yhdistellä olemassa olevia datamalleja siten,  
että haluttu lopputulos saavutetaan.  

Mitä kauemmas siirrytään selkeästä tehtävästä kuten objektin tunnistus,
hankaloituu totuuden määrittäminen ja näin ollen myös hyvän ja toimivan mallin luominen.

Tässä työssä käytetään sellaista luokittelu ja syvyysdataa mallin luomiseen joka antaa syvyysdatan
ilman liikkuvia kohteita, kuten autoja tai ihmisiä.
Tämä pyritään saavuttamaan mahdollisimman automatisoidusti  
ja tavalla, joka on hyödynnettävissä mahdollisimman geneerisesti erilaisen koulutusdatan kanssa.  

Lopullisen koulutusdatan tuottamiseen tarvitaan siis tieto irtonaisista kohteista kuvassa,  
kuvan syvyysdata sekä jonkinlainen arvio tunnistetun kohteen takana olevasta alueesta.  
Paras keino tämän datan arviointiin olisi kuvien manuaalinen läpikäynti ja syvyysdatan ”värittäminen” kuvaan.  
Tässä työssä kuitenkin pyritään tekemään työ mahdollisimman automaattisesti resurssien säästämiseksi.
Vaikka datan tuottamiseen voidaankin käyttää pitkälti perinteisiä kuvankäsittelyalgoritmeja, 
ei sen avulla todennäköisesti ole mahdollista tuottaa täysin luotettavaa dataa,  
koska kohteiden takana olevien asioiden tieto puuttuu stereokuvista.  
Tämän tiedon irrottaminen automaattisesti voisi olla mahdollista, jos lähtökohtana käytettäisiin videokuvaa,  
mutta tällöin toteutettavan mallin monimutkaisuus kasvaisi huomattavasti.  
Datan luominen olisi mahdollista myös ottamalla stereokameralla kuvia samasta kohteesta niin pitkällä aikavälillä, että kaikki liikkuvat kohteet olisivat väistyneet.  
Tällaisen datan luominen tai löytäminen on kuitenkin hyvin työlästä.

Nyt tehtyä mallia voisi pidemmälle jalostettuna hyödyntää moniin eri käyttökohteisiin, esimerkiksi automaatiossa, jossa tilasta vaaditaan puhdas 3D-malli.  
Näin voidaan tuottaa esimerkiksi automaattiajamisessa käytettävät liikkeet suoraan tyhjään tilaan.  
Samaa periaatetta voitaisiin hyödyntää tilojen 3D-skannauksessa,  
jolloin mahdollistuisi tilan skannaus ilman sen valmistelua tai datan jälkikäteistä manuaalista käsittelyä.  
Jatkokehityksessä tunnistetut huonekalut sekä muut siirrettävät esineet voitaisiin asettaa suoraan generoituun pistepilveen tai 3D-malliin.  
Tämä puolestaan mahdollistaisi 3D-skannauksen esimerkiksi peleihin tai muuhun interaktiiviseen sisältöön,  
jolloin ympäristö voisi olla suoraan interaktiivinen ilman manuaalista työtä.  

Tässä työssä käytettävä kuva- ja syvyysdata on itseajavien autojen kehitykseen liittyvää kaupunkidataa.  
Datasta löytyy valmiina sekä syvyysdata että segmentointidata.  
Koska mallin ja sen rakentamistyökalujen tulee olla hyödynnettävissä myös muulla datalla,  
käydään läpi myös stereokuvasta syvyysdatan hankinta sekä segmentointimallin luonti.  
Lopputuloksessa käytetään valmiiksi tarjottuja malleja segmentoinnin ja syvyysdatan osalta.  

Lopullinen työ on malli, jolle tarjotaan stereokuva ja se palauttaa syvyysdatan ilman liikkuvia kohteita kuvassa.  
