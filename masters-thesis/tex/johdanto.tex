\chapter{Johdanto}%
\label{ch:johdanto}

Koneoppimisella voidaan hakea ratkaisua monenlaisiin ongelmiin. Sen avulla voidaan pyrkiä ratkaisemaan normaalisti erittäin monimutkaisia algoritmeja vaativia ongelmia. Käytännössä mikä tahansa tietotekninen ongelma on teoriassa ratkaistavissa koneoppimisella. Se ei kuitenkaan aina ole tarpeellinen tai mahdollinen ratkaisu. Tarpeettoman monimutkaisuuden sekä ”black box” ongelmien välttelyn ohella, yhtenä suurimpana esteenä sen toteuttamisessa on hyvän koulutusdatan hankkiminen. Tämänhetkiset käyttökelpoiset mallit koulutetaan koulutusdatan avulla, ja nämä mallit voivat olla vain yhtä hyviä kuin niissä käytettävä koulutusdata.

Näin ollen jos halutaan luoda malli joka ei käytä yleisesti saatavilla olevaa koulutusdataa, hankaloituu sen luominen huomattavasti. Yleisimmin saatavilla olevat mallit ovat ihmisten sekä asioiden tunnistamiseen. Jos tavoittelemamme asia on variaatio tästä tai jostain muusta yleisestä mallista, voi uuden datan luomiseen alusta aloittamisen sijaan käyttää olemassa olevia malleja. Kuitenkin mitä kauemmas siirrytään selkeästä tehtävästä niin kuin objektin tunnistus, hankaloituu totuuden määrittäminen ja näin ollen myös hyvän ja toimivan mallin luominen.

Tässä työssä pyritään käyttämään segmentointi sekä syvyysdataa mallin luomiseen joka antaa syvyysdatan, ilman liikkuvia kohteita niin kuin autoja tai ihmisiä. Tämä pyritään saavuttamaan mahdollisimman automaattisesti, ja tavalla joka on hyödynnettävissä mahdollisimman geneerisesti eri mallien kanssa.

Lopullisen koulutusdatan generointiin tarvitaan siis tieto irtonaisista kohteista kuvassa, kuvan syvyysdata sekä jonkinlainen arvio tunnistetun kohteen takana olevasta alueesta. Paras keino tämän datan generointiin olisi kuvien manuaalinen läpikäynti ja syvyysdatan ”värittäminen” kuvaan. Tässä työssä kuitenkin pyritään työ tekemään mahdollisimman automaattisesti, resurssien säästämiseksi. Vaikka datan generointiin voidaankin käyttää pitkälti perinteisiä kuvankäsittely algoritmeja ei sen avulla todennäköisesti ole mahdollista tuottaa täysin luotettavaa dataa, koska kohteiden takana olevien asioiden tieto puuttuu stereokuvista. Tämän tiedon irrottaminen automaattisesti voisi olla mahdollista jos lähtökohtana käytettäisiin videokuvaa, mutta toteutettavan mallin kompleksisuus kasvaisi huomattavasti. Datasetin luominen alusta olisi myös mahdollista jos stereokameralla otettaisiin kuvia samasta kohteesta niin pitkällä aikavälillä että kaikki liikkuvat kohteet olisivat väistyneet. Tällaisen datasetin luominen tai löytäminen on kuitenkin hyvin hankalaa. 

Tällaista mallia voisi pidemmälle jalostettuna hyödyntää moniin eri käyttökohteisiin. Sitä voisi hyödyntää automatisoinnissa, kun tilasta saadaan helposti puhdas malli. Näin voidaan generoida esimerkiksi automaattiajamisessa käytettävät liikkeet suoraan tyhjään tilaan. Samaa voisi hyödyntää tilojen 3d skannauksessa. Ei olisi niin tarpeellista siivota tilaa ja jatkojalostuksella tunnistetut huonekalut tai muut siirrettävät asiat voitaisiin sisustaa suoraan generoituun pistepilveen tai 3d malliin. Tämä mahdollistaisi myös 3d skannauksen esimerkiksi peleissä, jolloin ympäristö voisi olla suoraan interaktiivinen, ilman niin suurta manuaalista työtä. 

Tässä työssä käytettävä data on itseajavien autojen kehitykseen liittyvää kaupunkidataa. Siitä löytyy valmiina, syvyysdata sekä segmentointidata. Kuitenkin koska mallin ja sen rakentamistyökalujen tulee olla hyödynnettävissä myös muulla datalla, käydään läpi myös stereokuvasta syvyysdatan hankinta, sekä segmentointimallin luonti. Lopputuloksessa käytetään kuitenkin, valmiiksi tarjottuja malleja segmentoinnin sekä syvyysdatan osalta.

Lopullisen tuotoksen olisi tarkoitus olla malli jolle tarjotaan stereokuva ja se palauttaa syvyysdatan ilman liikkuvia kohteita kuvassa.