In Machine learning usually one of the hardest tasks is collecting sufficient data.
This thesis tries to create a model, which can be used for detecting true depth ignoring all moving objects from a stereo image.
Because datasets for this kind of task are not easily available, we try to create them using some more commonly available data as a basis.

Machine learning is commonly used for various kinds of object detection and segmentation.
Therefore there are many models available one can use for detecting unstationary objects.

Another required dataset consists of stereo images with depth information.
As stereoanalysis was common practice even before machine learning became common.
Hence there is no shortage of stereo datasets.

Missing data is depth data behind detected moving objects.
There are no easily available datasets providing this specific type of data.
Therefore trainingdata is being generated behind the objects using existing objects around the detected items.
For this task we take advantage of stereo images with static camera placement, which allows us to estimate depth in different areas of the image.
Then we apply different sweeps towards different depths to depth maps achieving linear depth changes instead of static objects.

Because the final data is very approximate, we can’t assume that the resulting model will be accurate.
However, even an inaccurate model can give an estimate of usability of machine learning for processing this kind of "low quality" data for useful end results.
