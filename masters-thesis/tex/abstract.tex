In Machine learning usually one of the hardest tasks is collecting sufficien data.
This thesis tries to create a model, which can be used for detecting true depth ignoring all moving objects from a stereo image.
Because datasets for this kind of task is not easily available, we try to create it using some more commonly available data as a basis.

Machine learning is commonly used for different kinds of object detection and segmentation. Therefore there is lots of models available that we can use for detecting unstationary objects.

Other data we need is stereo images with depth. As stereoanalysis was common practice even before machine learning became commonplace there is no shortage of those datasets.

Missing data is depth data behind detected moving objects. There is not easily available datasets providing this data. 
Therefore trainingdata is being generated behind the objects using existing objects around the detected items. 
For this task we take advantage of static camera placement in multiple images, which allows us to estimate depth in different areas of the image.
Then we can do different kinds of swipes towards different depths achiecing linear depth changes instead of static objects.

Because final data is very approximate, we can't assume that the achieved model will be accurate.
However even with inaccurate model we canestimate if machine learning could be used for processing this kind of "low quality" data for useful end results.
