In Machine learning usually one of the hardest tasks is collecting sufficient data.
This thesis tries to create a model, which can be used for detecting true depth ignoring all moving objects from a stereo image.  
Because datasets for this kind of task is not easily available, we try to create it using some more commonly available data as a basis.

Machine learning is commonly used for various types of object detection and segmentation.
Therefore, there are many models available that we can use for detecting unstationary objects.  

Another required dataset consists of stereo images with depth information.
Since stereo analysis was a common practice even before machine learning became commonplace, there is no shortage of such datasets.  

Missing data is the depth information behind detected moving objects. There are no easily available datasets providing this specific type of data.  
Therefore, training data is generated behind the objects using existing objects around the detected items.
For this task, we take advantage of a static camera setup with two images, allowing us to estimate depth in different areas of the image.
Then, we apply different kinds of depth sweeps, achieving linear depth changes instead of static objects.  

Because the final data is very approximate, we cannot assume that the resulting model will be accurate.  
However, even an inaccurate model can help us evaluate whether machine learning could be used to process this kind of "low-quality" data for useful end results.
