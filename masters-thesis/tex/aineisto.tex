\chapter{Aineisto}%
\label{ch:aineisto}

Työn toteuttamiseen käytettiin cityscapes datasettiä \cite{Cordts2016Cityscapes}.


Datasetti pitää sisällään dispariteettidatan, sekä segmentaatiodatan.
Data on autosta stereokameralla kuvattua.
Malli on suunnattu automatisoituun liikenteeseen, ja siinä on paljon erilaisia kaupunkikuvia teiden varsilta.
Tämä tuottaa hankaluuksia lopputuloksen kanssa, sillä malli toimii reaalimaailmassa vain kaupunkikuvissa.
Kuitenkaan yleisempää disparteetti sekä segmentaatio mallia ei järkevästi löytynyt. 
Tämän ongelman olisi voinut kiertää käyttämällä erillisiä malleja segmentointiin sekä dispariteettiin. 
Kahden erialisen mallin yhdistäminen tuo kuitenkin enemmän riskejä.
Kahden mallin tapauksessa, jouduttaisiin joko luottamaan segmentaatiomalliin datan valmistelussa, 
tai kaikki data joduttaisiin käymään manuaalisesti läpi.
