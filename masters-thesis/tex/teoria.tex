\chapter{Teoria}%
\label{ch:teoria}

\section{Stereo nalyysi}

\begin{figure}[h]
\centering
\pdftooltip{\includegraphics[width=\textwidth]{figures/stereo_depth.jpg}}{Stereo depth}
\caption{Stereo depth}
\label{fig:stereo}
\end{figure}

Stereo analyysillä kuvankäsittelyssä tarkoitetaan kahdesta samasta kohteesta otetusta kuvasta olevien yhtenäisyyksillä syvyyden analysointia.
Vaikka näitä tekniikoita voi varioida mihin tahansa kuviin jotka ovat samasta kohteesta,
tässä yhteydessä käytämme kuvia,
jotka ovat otettu samasta perspektiivistä siten että kuvat ovat horisontaalisesti vierekkäin.
Jos tiedetään kameroiden sijainti toisistaan tai jonkin pisteen etäisyys kamerasta voidaan myös kuvasta arvioida absoluuttinen etäisyys kameraan. 

Kun stereoparia analysoidaan ja tunnistetaan korreloiva piste molemmista kuvista,
voidaan laskea kuvien välinen dispariteetti \ref{fig:stereo}
Tämä tarkoittaa käytännöstä muutosta pisteen sijainnissa kuvien välillä.
Periaatteessa vastaavat pisteet voivat sijaita missä tahansa kuvassa.
Kuitenkin tässä tapauksessa jossa kuvat ovat tietyssä suhteessa toisiinsa,
voidaan olettaa stereoparin löytyvän x akselilta tai ainakin melko läheltä sitä.
Jos näin ei olisi jouduttaisiin kuvan jokaista pistettä vertaamaan jokaiseen pisteeseen toisessa kuvassa.
Tämä tapa tulee nopeasti hyvin kalliiksi.
Jos verrattava alue on myös hyvin pieni todennäköisyys tunnistaa useita samankaltaisia alueita kasvaa.
Jos alue on suuri tai vertailu suoritetaan liian tarkasti,
kasvaa todennäköisyys että vaihtunut kuvakulma on niin erilainen että sitä ei tunnisteta.
Tästä johtuen tämän ongelman ratkaisuun on kehitetty monia erilaisia tapoja.

TODO: Tähän voisi hakea lähteitä. Ihan vaan joku geneerinen stereo analyysi artikkeli.


\subsection{SGM Semi global matching}

Jotta stereo analyysi on mahdollista,
tulee kuvasta tunnistaa samat kohteet.
Yksi tapa tehdä tämä on Hircmuller:in SGM tekniikan avulla\cite{hirschmuller2005babel}.
Tämä tekniikka ottaa huomioon pikselin ja sen ympäröivien pikselien arvot etsiessään toisesta kuvasta vastaavaa arvoa. Tämä hakuoidaan tehdä kaavalla.

\begin{equation}\label{yht:SGM}
    E(d) = \sum_{p} D(p, d_p) + \sum_{q \in \mathcal{N}} R(p, d_p, q, d_q)
\end{equation}

Kaavassa \(D(p, d_p)\) summa funktio käy läpi kaikki kuvan pikselit ja vertaa niitä vertailupikseliin.
Tämä antaa perus arvon yksittäisen pikselin samankaltaisuudelle.

Tämän jälkeen pikselin ympäröiviä pikseleitä verrataan toisiinsa \(R(p, d_p, q, d_q)\).
koska saman asian ympäröivät pikselit tulisi olla melko samankaltaisia molemmissa kuvissa, 
Vaikka se onkin kuvattu hieman eri asennosta, voidaan tämän avulla arvioida onko kyseessä sama piste.

Tämän algoritmin implimentoi pythonin opencv kirjaston SGBM \cite{opencvsgbm},
jota varioitu käymään läpi pikseli joukkoja, eikä yksittäisiä pikseleitä.
Tämä nopeuttaa prosessointia huomattavasti.
Se myöskin suorittaa joitain suodatuksia parempien tulosten saavuttamiseksi.

\section{Neuroverkot}

- TODO: Parantele selitystä

Lopulliseen toteutukseen käytetään neuroverkkoa, neuroverkko on yleisesti kuva-analyysiin käytettävä tekniikka, jonka voi kouluttaa datalla niin kuvankäsittelyllisiä kuin muitakin ongelmia. Sen toiminta perustuu neuroneihin joita järjestetään verkkomaiseen rakenteeseen useisiin eri kerroksiin.

\begin{equation}\label{yht:neuroni}
    a = \sigma\left(\sum_i w_i x_i + b\right)
\end{equation}

Tämä on neuronin matemaattinen kaava,
se tuottaa ulostulonaan arvon \(a\) saamiensa syötteiden perusteella.
\(x_i\) on neuronin saama syöte.
\(w_i\) on neuronille annettu painoarvo.
\(b\) on neuronin bias arvo. \(\sigma\) on funktio joka muuttaa neuronin saavan armon välille 0,1.

Kun näitä neuroneita asetetaan eri kerroksiin siten että verkon sisääntulo on esimerkiksi valokuvan kokoinen, ja ulostulo on yhen neuronin ulostulo, voidaan verkolle syöttää kuvia esimerkiksi kissoista ja koirista. Kun näille kuville annetaan arvot 0 ja 1 kuvan aiheen mukaan, voidaan verkko kouluttaa tunnistamaan kissoja ja koiria. Koulutuksen aikana verkko muuttaa arvojaan \(w_i\) ja \(b\). Nämä arvot se saa yrittämällä erilaisia arvoja neuroneille. Kun verkkoa tämän jälkeen testataan on voidaan saaduista lopputuloksista valita paras. Tämän jälkeen tätä lopputulosta voidaan lähteä parantelemaan, testaamalla toimiiko suuremmat vai pienemmät arvot paremmin. Kun näitä kahta arvoa eri neuroneilla muutetaan saadaan paremmin toimiva neuroverkko.

Neuroverkko on siis vain kasa yksinkertaisia matemaattisia funktioita, joiden toimintaa vain tuurilla arvoidaan jotta saadaan haluttu lopputulos. Tätä tuuria pyritään parantamaan ohjaamalla verkon koulutusta. Tämä tapahtuu tappiofunktion (loss function) sekä takaisinvirtausalgoritmin (backpopagation algorithm) avulla. Loss functionin tehtävä on kertoa kuinka paljon saatu tulos eroaa halutusta. Esimerkki tapauksessamme tämä käytännössä katsoisi onko kissa kuvan arvo mikä sen pitäisi olla. Tämän työn lopputulos on verkko joka yrittää luoda kuvasta 3d pistekartan. Siinä tapauksessa siis neliösumma tai jokin muu tapa virheen tunnistamiseen olisi parempi. Kun virhe on tunnistettu verkkoa muokataan takaisinvirtaus algoritmin perusteella. Tähän ei ole yhtä parasta ratkaisua, vaan eri verkkojen ja käyttökohteiden tapauksessa eri algoritmit voivat tuoda huomattavasti parempia tuloksia.

"\url{https://en.wikipedia.org/wiki/Artificial_neural_network}"

\section{Semantic segmentation}

- TODO: Parantele selitystä

Semantic segmentation eli kuvan segmentointi on yksi yleinen käyttökohde neuroverkoille.
Sen avulla on helppo tehdä käytettäviä ja helposti hyödynnettäviä malleja.
Esimerkkejä käytöstä on esimerkiksi automatisoidussa liikenteessä erilaisten asioiden erottelu \ref{fig:labels}.
Tätä samaa teknologiaa voidaan käyttää useisiin eri applikaatioihin ja se on mahdollisesti yksi parhaita tapoja tarjota tietokoneelle mahdollisimman ihmismäinen tapa tunnistaa asioita kuvista. Näin voidaan yhdestä kuvasta tunnistaa erilaiset löytyvät asiat, esimerkiksi autot, ihmiset tiet puut jne. Samaan asiaan voisi käyttää myös object detection mallia, mutta tunnistustapa on hieman erilainen. Toisinkuin objektin tunnistus joka tunnistaa kuvan alueen missä on esimerkiksi ihminen, segmentointi malli tekee erottelun tarkemmin pikselitasolla.

\begin{figure}[h]
\centering
\pdftooltip{\includegraphics[width=\textwidth]{figures/stuttgart03.png}}{Cityscapes esimerkki kuva stuttgart03}
\caption[Tämä on lyhyt kuvateksti.]{Cityscapes datestin esimerkki segmentointi dataa, jossa kaupunki näkymän erilaiset tunnistattavat kohteet on merkitty eri väreillä.}
\label{fig:labels}
\end{figure}

Kun segmentointi mallin kanssa työskennellään se kuitenkin tuo hieman lisähaastetta datasetin vaatimuksiin koska kuvien pitää olla käsitelty tarkemmin kuin vain varustettuna alueilla missä on asioita. Tämä möyskin hieman muutta tappiofunktion käsittelyä. Koska mallin ei ole tarkoitus palautta pistejoukkoja vaan alueita, ei voida vain verrata ovatko tunnistetut pisteet lähellä koulutuspisteitä, vaan tulee tehdä jonkinnäköistä pinta-alaan perustuvaa analyysiä. 

"Kirjoita lisää"

"\url{https://en.wikipedia.org/wiki/Image_segmentation}"

\section{3d pisteavaruus}

- TODO: Parantele selitystä

Lopullinen haluttu tuotos on 3d  pisteavaruus. 3d pisteavaruus voidaan ajatella esitetyistä pisteistä joilla on syvyys arvo. Tässä tapauksessa kun pisteavaruus on tehty valokuvien pohjalta, voi myös saadun pisteavaruuden esittää kuvana \ref{fig:depth}. Muissa tilanteissa 3d pisteitä voi kuitenkin sijaita kaksiulotteisesta perspektiivistä katsottuna toistensa takana. Ei siis tule olettaa että kaikki 3d pistekartat on häviöttömästi esitettävissä 2d muodossa. 


\begin{figure}[h]
\centering
\pdftooltip{\includegraphics[width=\textwidth]{figures/leverkusen_000024_000019_disparity.png}}{Cityscapes esimerkki kuva leverkusen_000024_000019_disparity}
\caption[Tämä on lyhyt kuvateksti.]{Citysscapes datasetin syvyysdata esimerkki.}
\label{fig:depth}
\end{figure}

3d pistekartat ovat hyvin perusmuotoinen ja epäoptimoitu tapa käsitellä 3d dataa. Jos sitä haluttaisiin optimoida 3d mallin kannattaisi muuttaa kolmioiksi tai muiksi monikulmioiksi joiden perusteella tämä data käsiteltäisiin. Mutta koska edellämainitusti oma datamme voidaan esittää 2d syvyydellisenä kuvana, meillä ei ole tähän tarvetta.

    
"\url{https://en.wikipedia.org/wiki/Point_cloud}"

